\documentclass{article}
\def\ntitle {Paper reading}
\def\nauthor{ Student ID: 11028253 } % default author is Zhengbo Zhou
% \def\notes{ } % default is the submitted version
% \def\ndate{ } % default is today's date
% \def\needcrop{ } % crop for easy previewing
% \def\fancysec{ } % section font become fancier 
\input{kai.tex}
%% redefine \emph and \bf such that they are colored and can be easily
%% transformed back to normal \emph and \bf
\renewcommand{\emph}[1]{\textit{\color{purple} #1}}
\renewcommand{\bf}[1]{\textsf{\bfseries \color{purple} #1}}
\def\dd{\mathrm{d}}
\numberwithin{equation}{section}
\begin{document}
\maketitle
% \tableofcontents
\thispagestyle{firstpage} 
%%%%%%%%%%%%%%%%%%%%%%%%%%%% MAIN ARTICLE %%%%%%%%%%%%%%%%%%%%%%%%%%%%
\section*{Week1}
12
Risken(1992)~\cite{Risken1992} conclude that Fokker Placnk equation can be solved by simulation method, eigenfunction expansion, 
numerical integration, the variational method and matrix continued-fraction method.
The special case of Fokker Planck equation has an analytical solution (e.g.,linear problems, stationary problems with one varibale).
For numerical methods, Park(1996)~\cite{Park_1996} compare six finite difference methods(three fully implicit and three semi-implicit).

\section*{Week2}
\subsection*{Derivation of Fokker-Planck equation(Kolmogorov forward eq)}
Lucas and Prescott (1971) \cite{Lucas_1971} define the stochastic process $x(t)$ as the Brownian motion with drift:
\begin{equation}\notag
    \dd x = \alpha \dd t + \sigma \dd w
\end{equation}
where $\dd w$ is the increment of a Wiener process, $\alpha$ is called the drift parameter, and 
$\sigma$ is called the varaince parameter. We divide the time interval t into n discrete steps with the each length $\Delta t$. In each 
step, while $x$ have the probability $P$ increase by an amount $\Delta h$, the probability of decreasing $\Delta h$ is $q=1-p$. 
Additionally, to keep the variance of $(x_t - x_0)$ independent of the particular $\Delta t$, we set $\Delta h =\sigma \sqrt{\Delta t}$.
The probability density function for $x(t)$ is written as $P( x,t \mid x_0,t_0)$, where $x_0$ is the given intitial value of $x$. Thus we write 
\begin{equation}\notag
    Prob(a\leq x(t) \leq b \mid x(t_0)=x_0) = \int_{a}^{b} P( u , t \mid x_0,t_0) \dd u.
\end{equation}
We assume the probability distribution is stationary and $P(x)$ satisfy
\begin{equation}\notag
    P(x)=p P(x-\Delta h) + q P(x+\Delta h)
\end{equation}
From time $t-\Delta t$ to $t$, the $x(t)$ can reach the point $x$ by increasing from the point $x- \Delta h$ or decreasing from the 
point $x+\Delta h$:
\begin{equation}\label{1}
    P( x,t \mid x_0,t_0)= p P( x-\Delta h,t -\Delta t \mid x_0,t_0) + q P( x+\Delta h,t-\Delta t \mid x_0,t_0).
\end{equation}

Then we expand $P( x-\Delta h,t-\Delta t \mid x_0,t_0)$ in a Taylor series around $P(x,t \mid x_0,t_0)$:
\begin{equation}\notag
    P(x-\Delta h, t-\Delta t \mid x_0,t_0)=P(x,t\mid x_0,t_0)-\Delta h \frac{\partial P}{\partial t} - \Delta h \frac{\partial P}{\partial x}+ \frac{1}{2}(\Delta h)^2
    \frac{\partial^2 P}{\partial x^2} + \cdots
\end{equation}
Since the third order terms and the higher order are consisted of order $(\Delta t)^{3/2}$, $(\Delta t)^2$, etc., these terms will achieve zero faster
than $\Delta t$. Then we expand $P(x+\Delta h, t-\Delta t \mid x_0, t_0)$ similarly and substitue these expressions into \eqref{2}:
\begin{align}\notag
    P(x,t\mid x_0, t_0)&=(p+q)P(x,t\mid x_0,t_0)-(p+q)\Delta t \frac{\partial P}{\partial t}\\
                       &-(p-q)\Delta h \frac{\partial P}{\partial x}+\frac{1}{2}(p+q)(\Delta h)^2 \frac{\partial^2 P}{\partial x^2}.
\end{align}

Using the discrete-time random walk representaion in Section 2.B, we substitue $p+q=1$, $p-q = (\alpha / \sigma)\sqrt{\Delta t}$ and $\Delta h = \sigma \sqrt{\Delta t} $. Then divide
by $\Delta t$, and rerrange:
\begin{equation}\label{2}
    \frac{1}{2}\frac{\partial^2}{\partial x^2} P(x,t \mid x_0, t_0) - \alpha \frac{\partial}{\partial x}P(x,t \mid x_0, t_0)= \frac{\partial}{\partial t}P(x,t \mid x_0,t_0).
\end{equation}
Equation \eqref{2} is called the Kolmogorv forward equation for the Brownian motion with drift. More general Kolmogorov forward equation for Ito process can be find in Karlin and Taylor(1981).

\subsection*{Evolution of Black-Scholes}
In the option pricing literature, Black and Scholes(1973)\cite{Black_1973} price the derivatives securities with the underlying stock price follows a 
geometric Brownian motion. It implies that the stock price obeys a lognormal disttribution and the implied volatility of 
options should be a constant with different time, strike prices and maturities. However Merton(1976)\cite{Merton_1976}, Christie(1982)\cite{Christie_1982}, Johnson and Shanno(1987)
\cite{Christie_1982} discuss variance changes instead of being a constant when price options. Use contents from \cite{AtSahalia2007} 
\cite{Christie_1982} \cite{Johnson_1987} to discuss the volatility should be changed to stochastic volatility.

\subsection*{QUAD method Andri 2003}
Andricopoulos et al. ~\cite{Andri_2003} conclude that Latice(binomial and trinomial trees), finite-difference, and Monte Carlo techniques are three categories to 
value derivatives.(especially powerful in pricing the path-dependent options). However, there are two main erros: "distribution error" and "nonlinearity error". 
To reduce distribution error, finer lattice or gird can combat the source of distribution erro, which comes with a more and more computing time in each refinement.
 By quadrature (QUAD) method, a full range of final position can be evaluated using one timestep, which improves the effiency. For reducing nonlinearity error in 
 lattice and finite-difference methods, there are more and more sources of nonlinearity error at different values of the underlying at different time. The 
 inflexibility of these approaches can not be cope. With QUAD, noelinearity can be avoided by positioning nodes exactly as required. It can exactly place "nodes"
 on boundaries or discontinuities and only a single timestep is required between observations of exotic features.

 

 \subsection*{Andri 2007}
 Andricopoulos et al. ~\cite{Andri_2007} extend the quadrature method to price options involving multiple underlying assets. The avoiding of nonlinearity error 
 makes QUAD method outperfoms Monte Carlo method in effiency and accuracy when pricing complex path-dependent options, especially with early exercise features.
 \subsubsection*{Early version of QUAD method}

 The QUAD method as a felxible, robust option pricing numerical technique. In the 
 Black-Scholes-Merton framemwork, the option price $V(S,t)$ obeys the geometric Brownian motion:
 \begin{equation}\notag
     \frac{\partial V}{\partial t}+ \frac{1}{2}\sigma^2 S^2 \frac{\partial^2 V}{\partial S^2}+(r-D_c)S \frac{\partial V}{\partial S} - rV =0,
 \end{equation}
 where $V$ is the price of derivatives securities and $S$ is the value of the underlying asset, $t$ is the time, $r$ is the risk-free interest
 rate. $\sigma$ is the volatility of the underlying asset, and continuous dividend is $D_c$. When the QUAD method is adapted to Black-Scholes-
 Merton framemwork, it is convenient to make a lo transform of the underlying asset in the above eq:
 \begin{equation}\notag
     x=log S
 \end{equation}
 Suppose $y$ is the log transformation of underlying asset at time $t+\delta t$, where $\delta t$ is the time step and it is not restricted to 
 small time periods. The payoff function is transformed to be $\max(e^y-X,0)$, where $X$ is the strike price. Then the option price is adapted
 to be :
 \begin{equation}\label{3}
     V(x,t)=A(x)\int_{-\infty}^{\infty}B(x,y)V(y,t+\delta t)dy,
 \end{equation}
 \begin{equation}\notag
     A(x)=\frac{1}{\sigma \sqrt{2\pi \delta t}}exp\left(-\frac{kx}{2}-\frac{\sigma^2 k^2 \delta t}{8}-r\delta t\right)
 \end{equation}
 and
 \begin{equation}\notag
     B(x,y)=exp\left(\frac{-(x-y)^2}{2\sigma^2 \delta t}+\frac{ky}{2}\right),
 \end{equation}
 and 
 \begin{equation}\notag
     k=\frac{2(r-D_c)}{\sigma^2} -1
 \end{equation}
 Through integration\eqref{3} many quadrature methods like Simpson's rule and the trapezium rule can be applied 
 as a valuation engine. For example, for a European call option, the integrand becomes 
 \begin{equation}\notag
     f(x,y)=B(x,y)max(e^y-X,0).
 \end{equation} 
 Then, after construction of QUAD grid with discontinuities(strike,earoy exercise or boundaries), the result can be obtained
 by Richardson extrapolation. The result only coincides with distribution error.

\subsection*{Chen 2014}
Mathematically, the QUAD method in \cite{Andri_2007} is a application of
Green's function. Under the Black-Scholes-Merton 
framemwork, Chen(2014)~\cite{Chen_2014} extend QUAD by using the closed-form approximations for transition density functions, where 
the Green's function are not available. Then they reformulate the value function in terms of probability density functions.
The asset price model is 
\begin{equation}\label{4}
    V(x,t)=e^{-r \tau} \int_{-\infty}^{\infty} V(y,T)f_\tau(y|x)dy,
\end{equation}
where $V$ is the value function of the option with underlying asset $x$ at this time $t$, and $f_\tau(y|x)$ is the risk-neutral 
condition density function of the asset price over time step $\tau =T-t$. The equivalence with the QUAD method in \cite{Andri_2003} is 
the Black-Scholes-Merton setup the density as 
\begin{equation}\label{5}
    f_\tau(y|x)=\frac{1}{y\sigma \sqrt{2\pi \tau}}exp\left(-\frac{1}{2}\left(\frac{\ln y - \ln x + (r-1/2 \sigma^2) \tau}{\sigma \sqrt{\tau}}\right)^2\right)
\end{equation}
The above equation is also mentioned in Su and Newton(2020)~\cite{Su_2020}. The difficulty is to computing the values of transition density $f_\tau (y|x)$ of 
the log-asset price. This method overcoming the limitation in pricing options under stochastic volatility models such as Heston's SABR (In section 5.2, they 
show how to reproduce volatility smiles by fully stochastic volatility 
models - SABR). Then they describe methods of approximating the density function by A\"{i}t-Sahalia's algorithm, the constant elasticity of variance(CEV) 
process and quadratic local volatility.

\subsection*{Su and Newton 2020}
Su and Newton(2020)\cite{Su_2020} try finite difference method for approximating transition density function. And combine the numerical method with QUAD method in \cite{Chen_2014}.
In this paper, they state that the density function of stochastic process is following the Kolmogorov forward and backward equation. Also, they use SABR model to produce the 
implied volatility smiles in the reality. 

They do not care about wether the transition probability density function has a closed-form expression. They approximate the density by solving the governing partial differential equations
of the transition density function. 


\subsection*{Question}
1. Should I show the general derivation of Kolmogorov forward equation? I do not find the proof in Karlin and Taylopr (1981). But I will try it agiagin, maybe could you recommend a
reference?

2.How the value fuction of option \eqref{3} and the value fuction of option \eqref{4}  come? In Su and Newton (2020) \cite{Su_2020} the density function \eqref{5} is obtained by solving the 
Kolmogorov forward equation with intitial conditon obeying Dirac delta funciton. \eqref{5} also represent in the \cite{Chen_2014}. I am confused. Is the density function a trivial thing? Can 
we easily solve the parabolic PDE?

3.Does the transition density function in option pricing only governed by Kolmogorov equation? The idea is stated by Su and Newton (2020)\cite{Su_2020}.

4.Idea of my key points about my topic: Option pricing can use the transition density and the Fokker Planck equation(FPE) can provide the source. 
Then, there is many way to solve FPE, I will try the finite difference method like Crank–Nicolson method. Besides QUAD method provide a powerful way to price complex option.
Hence, I will combine finite difference method and QUAD method. In terms of the Blakc-Scholes framemwork, changing constan volatility to local volatility(CEV) or to stochastic volatility (SABR model)
can produce smile volatility. 
I will first try use these method in pricing a European option and the turn into pathdepent option. But I have not a clear framemwork of my dissertation. Could you please give me some advice?

5.When writing the literature review, I will first find the relevant paper that derivartion of the FPE and the link between this with option pricing. But I do not find clear paper that talking about it.
Then compare the difference of price option by constant volatility, local volatility and stochastic volatility.Then I will conclude the methods to solve FPE when it can provide a source to derive transition density. 
While conclude the improvment of QUAD method, I also compare other methods also derive the density function (Maximum likelihood \cite{AtSahalia2007}). What I need to focus more? Could you please give me some advice?

\bibliographystyle{plain}
\bibliography{/Users/zk/Documents/bibtex/ref.bib}
\end{document}